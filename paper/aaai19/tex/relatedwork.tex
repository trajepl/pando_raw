\section{Related Work}
\label{related-work}
Semantic measures are mathematical tools used to estimate the intensity of the 
semantic relationship between units of language, concepts or instances, through 
a numerical description. Many traditional studies on semantic relatedness utilize
different data resources to compute semantic relatedness. There are
% The semantic measures contain semantic relatedness,
% semantic similarity and semantic distance. The semantic distance equals semantic 
% unsimilarity. The similarity is only a particular case of relatedness.
% There are so many researches \cite{acl/IacobacciPN15}, \cite{tkde/LiBM03}, \cite{tkde/ZhuI17}
% which study the semantic similarity, and make significant accomplishment, even so,
% comparison for similarity fails to give a complete view of relatedness measures.

i) \emph{the large corpora}, such as wikipedia. 
% WikiRelate! \cite{aaai/StrubeP06} and  Explicit Semantic Analysis(ESA)\cite{ijcai/GabrilovichM07} exploited
% texts in the pages and wikipedia categories to compute semantic relatedness.
The initial model WikiRelate! \cite{aaai/StrubeP06}, firstly retrieved the corresponding Wikipedia
articles whose titles contain the words in input and estimated relatedness based on strategies in articles of wikipedia.
Explicit Semantic Analysis(ESA)\cite{ijcai/GabrilovichM07} represented the meaning of texts in a high-dimensional space
They preprocessed the content of Wikipedia to build an inverted index for each word in texts.
Relevance was computed by the TFIDF weighting scheme while relatedness was computed by
the cosine of the vectors associated to the texts. WikiRelate! and ESA only leveraged texts in
Wikipedia and did not consider links among articles. Another model WLM \cite{aaai/Milne08} scrutinized incoming/outgoing
links to/from articles instead of exploiting texts in Wikipedia articles.
WikiWalk \cite{textgraphs/YehRMAS09} extended the WLM by exploiting not only links that appeared in an article
(i.e., a Wikipedia page) but all links, to perform a random walk based on Personalized PageRank.

% ii) \emph{the lexical databases}, such as WordNet or Wikithionary. 
% In the wordnet-based methods\cite{acl/Pucher07}, computed semantic relatedness for automatic speech recognition for meetings.
% This work did not provide a individual result to reveal the efficiency of semantic relatedness measures.
% The paper \cite{aaai/ZeschMG08} introduced Wikithionary as an emerging lexical semantic resource
% that could be used as a substitute for expert-made resources in AI applications.
% They chose(1) a path based approach\cite{its/Rada89}, which can be utilized with any
% resource containing concepts connected by lexical semantic relations. (2) a concept vector based approach
% \cite{ijcai/GabrilovichM07}. They generalize this approach to work on each resource which offered a textual representation of a concept.

% iii) \emph{the knowledge graph}. 
% With the increasing popularity of the linked data, many public Knowledge Graphs (KGs) have
% become available, such as DBpedia and BabelNet which are novel semantic networks recording millions
% of concepts, entities and their relationships.
% Recently, many researchers have used the Knowledge Graph as background knowledge to compute semantic relatedness.
% In BabelRelate\cite{aaai/NavigliP12}, they presented a knowledge-rich approach to compute multilingual semantic
% relatedness which exploited the joint contribution of different languages. Given a pair of words in two languages,
% The SensEmbed \cite{acl/IacobacciPN15} leveraged BabelNet\footnote{http://babelnet.org} to annotate the dump of wikipedia,
% and exploitd word2vec\cite{corr/Mikolov13} train the sense-annotated wikipedia to get distributed representation of different 
% word senses. Essentially this method is based on \emph{the large corpora} and needs a significant preprocessing
% and data transformation efforts. 
% The REWOrd\cite{aaai/Pirro12} proposed an approach that exploited the graph nature of RDF and SPARQL query
% Language to access knowledge graph. It not only obtained the comparable result with the state-of-art at that moment,
% but also avoids the burden of preprocessing and data transformation.
% However, REWOrd lost sight of the informativeness of the other entities, while they just
% considered the entity with the highest rank associated with the input words.
% Secondly, it missed some informativeness of \emph{predicates} as their strategy took
% the \emph{predicates} into account exclusively based on the TFIDF, which ignored the function of \emph{objects} in a semantic triple.

% In summary, in order to improve the performance of DBpedia-based model especially the REWOrd,
% We consider multiple entities associated with input words rather than just one with hightest rank.
% Then we utilize the method of embedding to generate high-dimensional vector for corresponding entities.
% For an input pair of words, we get two sets of corresponding entities in DBpedia ont only the one with highest rank,
% then get multiple relatedness scores after full links between two sets of entities.
% In order to better fit the judgement of human, we utilize a combinatorial strategy to combine
% the relatedness scores of pairwise entities.

% SensEmbed\cite{acl/IacobacciPN15} leveraged entity linking to annotate the dump of wikipedia. Based on this,
% the sense-annotated corpus was generated. Then the author used word2vec to
% train the sense-annotated corpus and get distributed representations of different 
% word senses. This step still needed a significant preprocessing and data transformation effort. 
% As we can see that this approach computes semantic relatedness on the strength of large corpora.
% The author just regarded the knowledge graph as a tool which was utilized to annotate the large corpora, wikipedia.