\section{Related Work}
Semantic measures are mathematical tools used to estimate the strength of the 
semantic relationship between units of language, concepts or instances, through 
a (numerical) description obtained according to the comparison of information 
supporting their meaning. The semantic measures contain semantic relatedness,
semantic similarity, semantic distance. The semantic distance equal to semantic 
unsimilarity. The similarity is only one particular type of relatedness.
There are so many researchs \cite{acl/IacobacciPN15, tkde/ZhuI17, tkde/LiBM03}
which study the semantic similarity, and make significant accomplishment, even so,
comparison to similarity fails to give a complete view of a relatedness measures.

% In this paper we focus on computing semantic relatedness on the account of knowledge
% graph. 
Many traditional studies on semantic relatedness
utilize different data sources to compute semantic relatedness. There are

i) \emph{the large corpora}, such as wikipedia. 
WikiRelate! \cite{aaai/StrubeP06}, given two words
first retrieves the corresponding Wikipedia articles whose titles
contain the words in input. Then, it estimates relatedness
according to different strategies among which comparing
the texts in the pages or computing the distance between
the Wikipedia categories to which the pages belong.

Explicit Semantic Analysis(ESA) \cite{ijcai/GabrilovichM07} makes a great impovement for computing semantic relatedness. 
ESA represents the meaning of texts in a high-dimensional space of concepts derived from Wikipedia.
They preprocess the content of Wikipedia to build an inverted index that for each word in texts.
Relevance is assessed by the TFIDF weighting scheme while relatedness is computed by
the cosine of the vectors associated to the texts.
Besides, WLM \cite{aaai/Milne08} instead of exploiting text in Wikipedia articles, 
scrutinizes incoming/outgoing links to/from articles. WikiWalk \cite{textgraphs/YehRMAS09} extends the
WLM by exploiting not only link that appear in an article (i.e., a Wikipedia page) but all 
links, to perform a random walk based on Personalized PageRank.

As the correlation coefficient in experiments show, the most promising approach is ESA in semantic 
relatedness measure using wikipedia. 
However, ESA requires a huge preprocessing effort to build the index for each word,
and only leverages text in Wikipedia and does not consider links among articles. ESA is Ineffective.
Besides, it may suffer some problems when changing the source of background knowledge.

ii) \emph{the lexical databases}, such as WordNet or Wikithionary. 
In the wordnet-based methods\cite{acl/Pucher07}, the relatedness of a word and a context is
defined as the average of the relatedness of the word and all words in the context. They compute
semantic relatedness for n automatic speech recognition for meetings, do not provide a individual result
to reveal the efficiency of semantic relatedness measures.
\cite{aaai/ZeschMG08} introduce Wikithionary as an emerging lexical semantic resource
that can be used as a substitute for expert-made resources in AI applications.
Then they evaluate Wikithionary on the task of computing semantic relatedness.
They choose(1) a path based approach\cite{its/Rada89}, can be utilized with any
resource containing concepts connected by lexical semantic relations. (2) a concept vector based approach
\cite{ijcai/GabrilovichM07}. They generalize this approach to work on each
resource which offers a textual representation of a concept.

iii) \emph{the knowledge graph}. 
With the increasing popularity of the linked data initiative, many public Knowledge Graphs (KGs) have
become available, such as DBpedia, BabelNet which are novel semantic networks recording millions
of concepts, entities and their relationships. Rencently, researchs have used the Knowledge Graph as
background knowledge to computing semantic relatedness. 

\cite{acl/IacobacciPN15} leverage entity linking to annotate the dump of wikipedia. Based on this,
the sense-annotated corpus is generated. Then the author using word2vec to
train the sense-annotated corpus getting distributed representation of different 
word sense. This step still need a significant preprocessing and data transformation effort. 
As we can see this approache computing semantic relatedness on account of large corpora.
The author just put the knowledge graph on the position of support. 

\cite{aaai/NavigliP12} presents a knowledge-rich approach to computing multilingual semantic
relatedness which exploits the joint contribution of different languages. Given a pair of words 
in two languages they use BabelNet to collect their translations, compute semantic
graphs in a variety of languages, and then combine the empirical evidence from these 
different languages by intersecting their respective graphs.

\cite{aaai/Pirro12} propose an approache exploiting the graph nature of RDF and SPARQL query
Language to access knowledge graph. It not only obtains the comparable
result with the state-of-art at that moment, but also avoids the burden
of preprocessing and data transformation.
Though \cite{aaai/Pirro12} avoids a significant preprocessing and data
transformation effort, which develops it scalability when adopting a
different source of knowledge graph, they miss some factors which
have contributed to semantic relatedness measure. Firstly, given two words
as input, the first step is to find corresponding entities in knowledge
graph. Obviously, there are not only one entity in knowledge graph for a single
word. For a input word \emph{car}, for example, we will get \emph{:Automobile} and
\emph{:Auto\underline{\hspace{0.5em}}Racing} and so on. \cite{aaai/Pirro12} lose
sight of the informativeness of the other entities, while they just
consider the entity with the highest rank. Secondly, \cite{aaai/Pirro12} misses
some informativeness of \emph{predicates} as their stategy takes
the predicates into account exclusively based on the TFIDF.This
method ignore the function of \emph{objects} in a semantic triple.

In summary, We use the Knowledge Graph(DBpedia) which derived from wikipedia as the background knowledge.
Then we use knowledge graph embedding to generating high-dimensional vector representing corresponding
entities. For a input pairwise word, we get two sets of corresponding entities in DBpedia, resulting multiple
relatedness scores after a join between two set of entities.
In order to better fitting the judgement of human, we propose a combinatorial strategy to combine
multiple the relatedness scores of pairwise entities.