\section{Related Work}
\label{related-word}
Semantic measures are mathematical tools used to estimate the intensity of the 
semantic relationship between units of language,concepts or instances, through 
a numerical description according to the comparison of information 
supporting their meaning. The semantic measures contain semantic relatedness,
semantic similarity and semantic distance. The semantic distance equals semantic 
unsimilarity. The similarity is only a particular case of relatedness.
There are so many researches \cite{acl/IacobacciPN15}, \cite{tkde/LiBM03}, \cite{tkde/ZhuI17}
which study the semantic similarity, and make significant accomplishment, even so,
comparison for similarity fails to give a complete view of relatedness measures.
Many traditional studies on semantic relatedness utilize different data resources
to compute semantic relatedness. There are

i) \emph{the large corpora}, such as wikipedia. 
In WikiRelate! \cite{aaai/StrubeP06}, for given two words
they first retrieved the corresponding Wikipedia articles whose titles
contain the words in input. Then they estimated relatedness by
employing different strategies among which comparing
the texts in the pages or computing the distance between
the Wikipedia categories to which the pages belong.

Explicit Semantic Analysis(ESA)\cite{ijcai/GabrilovichM07} made a great improvement for computing semantic relatedness. 
ESA represented the meaning of texts in a high-dimensional space of concepts derived from Wikipedia.
They preprocessed the content of Wikipedia to build an inverted index for each word in texts.
Relevance was computed by the TFIDF weighting scheme while relatedness was computed by
the cosine of the vectors associated to the texts.
Besides, WLM \cite{aaai/Milne08} scrutinized incoming/outgoing links to/from articles instead of
exploiting texts in Wikipedia articles.
WikiWalk \cite{textgraphs/YehRMAS09} extended the
WLM by exploiting not only links that appeared in an article (i.e., a Wikipedia page) but all 
links, to perform a random walk based on Personalized PageRank.

As the correlation coefficient in experiments show, the most promising approach to semantic relatedness
is ESA which used wikipedia. 
However, a huge preprocessing effort are required when it comes to ESA for building the index for each word.
ESA only leveraged texts in Wikipedia and did not consider links among articles. ESA was ineffective.
Besides, it might suffer some problems when changing the resource of background knowledge.

ii) \emph{the lexical databases}, such as WordNet or Wikithionary. 
In the wordnet-based methods\cite{acl/Pucher07}, the relatedness of a word in a context was
defined as the average of the relatedness of the word and all words in the context. They computed
semantic relatedness for automatic speech recognition for meetings. This work did not provide a individual result
to reveal the efficiency of semantic relatedness measures.
The paper \cite{aaai/ZeschMG08} introduced Wikithionary as an emerging lexical semantic resource
that could be used as a substitute for expert-made resources in AI applications.
Then they evaluated Wikithionary on the task of computing semantic relatedness.
They chose(1) a path based approach\cite{its/Rada89}, which can be utilized with any
resource containing concepts connected by lexical semantic relations. (2) a concept vector based approach
\cite{ijcai/GabrilovichM07}. They generalize this approach to work on each
resource which offered a textual representation of a concept.

iii) \emph{the knowledge graph}. 
With the increasing popularity of the linked data, many public Knowledge Graphs (KGs) have
become available, such as DBpedia and BabelNet which are novel semantic networks recording millions
of concepts, entities and their relationships. Recently, many researchers have used the Knowledge Graph as
background knowledge to compute semantic relatedness. 

The paper\cite{acl/IacobacciPN15} leveraged entity linking to annotate the dump of wikipedia. Based on this,
the sense-annotated corpus was generated. Then the author used word2vec to
train the sense-annotated corpus and get distributed representations of different 
word senses. This step still needed a significant preprocessing and data transformation effort. 
As we can see that this approach computes semantic relatedness on the strength of large corpora.
The author just regarded the knowledge graph as a tool which was utilized to annotate the large corpora, wikipedia.  

In BabelRelate\cite{aaai/NavigliP12}, they presented a knowledge-rich approach to compute multilingual semantic
relatedness which exploited the joint contribution of different languages. Given a pair of words 
in two languages, they used BabelNet to collect their translations, computed semantic
graphs in a variety of languages, and then combined the empirical evidence from these 
different languages by intersecting their respective graphs.

The REWOrd\cite{aaai/Pirro12} proposed an approach that exploited the graph nature of RDF and SPARQL query
Language to access knowledge graph. It not only obtained the comparable
result with the state-of-art at that moment, but also avoids the burden
of preprocessing and data transformation.
Though they avoided the preprocessing and transformation effort,
which develops its scalability when adopting a different resource of knowledge graph,
they missed some factors which had contributions for semantic relatedness measurement. Firstly, given two words
as input, the REWOrd firstly found those corresponding entities in knowledge
graph. Obviously, there were usually not only one entity in knowledge graph for a single
word. For an input word \emph{car}, for example, we will get \emph{:Automobile} and
\emph{:Auto\underline{\hspace{0.5em}}Racing} and so on. The REWOrd lost
sight of the informativeness of the other entities, while they just
considered the entity with the highest rank. Secondly, it missed
some informativeness of \emph{predicates} as their strategy took
the \emph{predicates} into account exclusively based on the TFIDF,
which ignored the function of \emph{objects} in a semantic triple.

In summary, we use the Knowledge Graph DBpedia which derived from wikipedia to compute semantic relatedness.
Then we utilize the method of knowledge graph embedding to generate high-dimensional vector for corresponding
entities. For an input pair of words, we get two sets of corresponding entities in DBpedia, then get multiple
relatedness scores after full links between two sets of entities.
In order to better fit the judgement of human, we propose a combinatorial strategy to combine
the relatedness scores of pairwise entities.