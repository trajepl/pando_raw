\section{Methology}
In order to compute semantic relatedness of a word pair, we propose a model which is
threefold. 
For a given pair of words, we first query the corresponding entities in knowledge graph.
We need to construct a specific graph contains all related entities and attributes between the 
corrsponding entity pairs. Then we use \emph{Starspace} to train the constructed 
graph. For each entity and relationship, this method produce a representation of vector.
In our method, we use cosine function to compare vectors corresponding to word pairs to get the relatedness measure.
Besides, in the query step, we would get several corresponding entities for an input word. Inspired by
\cite{acl/IacobacciPN15}, we combine the relatedness scores computing from the multiple pairs of entities 
as the final measure sorce between two words in knowledge graph.

% Please add the following required packages to your document preamble:
% \usepackage{booktabs}
\begin{table*}[]
    \small
    \centering
    \caption{Query Entity}
    \label{entities}
    \renewcommand\arraystretch{1.6}
    \setlength{\tabcolsep}{0.8mm}{
    \begin{tabular}{llclc}
    \toprule
    \textbf{Words}  & automobile                                               & \textbf{Refcount} & car                                      & \textbf{Refcount} \\ \hline
    \textbf{Entity} & http://dbpedia.org/resource/Automobile                   & 5096     & http://dbpedia.org/resource/Automobile   & 5096     \\ \hline
    \textbf{Entity} & http://dbpedia.org/resource/Auto\_racing                 & 2885     & http://dbpedia.org/resource/NASCAR       & 4317     \\ \hline
    \textbf{Entity} & http://dbpedia.org/resource/Ferry                        & 2501     & http://dbpedia.org/resource/Tram         & 3988     \\ \hline
    \textbf{Entity} & http://dbpedia.org/resource/Internal\_combustion\_engine & 1639     & http://dbpedia.org/resource/Auto\_racing & 2885     \\ \hline
    \textbf{Entity} & http://dbpedia.org/resource/Gasoline                     & 1486     & http://dbpedia.org/resource/Ferry        & 2501     \\ \hline
    \end{tabular}    
    }
\end{table*}

\subsection{Construct graph}
\label{sec:construct graph}
Our aim is to compute the semantic relatedness between a pair of words. The process of relatedness measure
needs complete and ample background knowledge which can be gathered in knowledge graph, such as
DBPedia\footnote{http://wiki.dbpedia.org/}, 
YAGO\footnote{https://www.mpi-inf.mpg.de/departments/databases-and-information-systems/research/yago-naga/yago} etc.
The first problem we face is how to obtain knowledge which is associated with given words from knowledge graph. 
In our model, we utilize the DBPedia as our knowledge base to gather corresponding entities triggered by the given words.
We relies on lookup services that is provided by DBPedia \footnote{http://lookup.dbpedia.org/api/search/KeywordSearch}. 
We use $W_{w_i}$ to denote the set of corrsponding entities in knowledge graph when we complete a query by word $w_i$,
i.e., $W_{w_i}=\{E_1,...,E_k\}$, $E_i$ is the $i_{th}$ query entity.
As show in Table \ref{entities}, for a given word pair \emph{automobile and car}, 
we can get the corresponding entities which are described as URIs for in DBPedia. 
These URIs describe entities so accurate that we can access knowledge graph using powerful query language SPARQL 
to get everything we want.

An entity is not only surrounding by entities, but also attributes. Let us see a example to explain the attributes 
in knowledge graph. There is a person A, his age is \emph{24}. He is a \emph{male}.
The number \emph{24} and the literal \emph{male} are not entities in knowledge graph,
but all are the attributes surrounding this person.
We can use the query results to access knowledge graph after the above step is done. Next we need to construct graph 
contains all related entities, attributes and relations between the corrsponding entity pairs. Inspired by \cite{aaai/NavigliP12},
we propose a improved method to get our semantic subgraph. We get query entities $W_{w_1}$ and $W_{w_2}$ individually
corresponding to words $w_1$ and $w_2$. 
Then we start by selecting the subgraph of DBPedia which contains all the path between entities
($ENT = W_{w_1} \cup W_{w_1}$) and all attributes for each each entity in $ENT$. We do 
this by building a directed grah ${G = (V, E)}$ which contains all relevant information which describes
the entities set.

i) We first define the set $V$ in $G$: $V:=ENT$. 
The size of set $V$ is not fixed. It would be extended in the following steps.
As for the set $E$, we initialize it as empty, i.e., $E:=\emptyset$.

ii)The goal of our method is to get the precise vector representation for corresponding entities,
that acquires more complete information surrounding the entities.
Accordingly, we can not only consider the neighbor entities of corresponding entity, but also
need to find all path connecting the nodes in $V$. 
We firstly get the one step neighbors for each $v \in V$. It is known to us all, the shorter
length of path between two entities, the more relative they are.
Secondly, we adopt Depth-First Search(DFS) to go through the knowledge graph. Every time we find a node
$v^{'} \in V$ but $v \ne v^{'}$ along a path($v, v_1, v_2,...,v_n, v^{'}$), we add all intermedia 
nodes and edges in this path to $G$, i.e., $V:=V \cup \{v_1, ..., v_n\}$, 
$E:=E \cup \{(v, v_1), ..., (v_k, v^{'})\}$.

iii) Next, we get all relevant attributes which are described as literal, number or something 
else special sympol in knowledge graph. For each $v \in V$, we collect all the surrounding attributes 
$\{a_1, a_2, ..., a_k\}$. Then we have $V:=V \cup \{a_1, ..., a_k\}$, 
$E:=E \cup \{(v_i, a_1), ..., (v_i, a_k)\}$ ($v_i \in V$).

By this way, we extract a subgraph from DBPedia which consists of the relevant information which describes
the entities set.

\subsection{Embedding for Subgraph}
\label{sec:train}
The constructed graph is fundamentally a multi-relational graph in which a entity is described by a set of discrete
\emph{entities and attributes}. Fortunately, there have been a excellent work proposed by Facebook AI Research
, \emph{StarSpace}. The model works by embedding those entities comprised of discrete featrues and
comparing them against each other. In this section, we introducte the basic contents of \emph{StarSpace} briefly and
how we utilize this model to get the vector representation of entities and relations. \emph{StarSpace} is available as
an open-source project at \url{https://github.com/facebookresearch/StarSpace}.

In \emph{StarSpace}, to train our model, we need to compare entities which is described by a set of discrete
\emph{entities and attributes}. Specially, there is the following loss function in $StarSpace$:

\begin{small}
    \begin{equation}
        \nonumber
        \label{starspace_formula}
        \sum_{\substack{(a,b) \in V^+\\ b^- \in V^-}}L^{batch}(sim(a,b),sim(a,b_1^-),...,sim(a,b_k^-))
    \end{equation}
\end{small}

In our problem, the input data is a graph of $(h, r, t)$ triples, consisting of a head entity $h$, 
a relation $r$ and a tail entity $t$.
Following the original paper which describes $StarSpace$, there are several explanations for this loss function:

1) The positive entity pairs (a,b) come from the set $V^+$ sampled from constructed graph $G$. 
In our problem, the input is a group of triples $(h, r, t)$. So how to make our input fit
the sample batch $(a, b)$? To this end, we need to select uniformly at random either to
get positive sample $V^+$ in two strategies:
(i)$a$ consistes of the bag of features $h$ and $r$, while $b$ consistes only of $t$; 
(ii)$a$ consistes of $h$, and $b$ consists of $r$ and $t$. 

2) Negative entities $b^-$ are sampled from the set of possible entities $V^-$.  
StarSpace utilize a $k$-negative sampling strategy\cite{corr/Mikolov13} to select $k$ negative pairs for each batch update. 
They select randomly from whithin the set of entities that can appear in the second argument of the similarity function.

3) The selection of function $sim(.,.)$ is designed as a hyperparameter: cosine similarity and inner product.
In our problem, we adop cosine similarity for the model as the cosine works better than inner product for
larger numbers which is mentioned in the paper of StarSpace.

4) The loss function $L_{batch}$ which compares the positive pair $(a,b)$ with the negative pairs $(a, b_i^-)$, $i=1,...,k$.
It is also optional between margin ranking loss and negative log loss of softmax. All experiments in $StarSpace$ show
the former performed on par or better. Thus we use margin ranking loss as our loss function for computing semantic relatedness.

5) The method optimazation inherit the stochastic gradient descent(SGD) used in $Starspace$. Each SGD step is one
sampled from $V^+$ in the outer sum.

As a result, we take the constructed graph $G$ of $(h, r, t)$ triples as inputs for the training model.
For each entity and relation in graph $G$, there is a fixed-length vector as the learnt embedding which
can then be used to compute semantic measure via cosine function.

(Experiments:K-sample,dim of vector,epochs)
% \cite{aaai/BordesWCB11}


\subsection{Semantic Relatedness Measure}
\label{sec:measure}
For a given word pairs($w_m$, $w_n$), we get ($W_{w_m}$, $W_{w_n}$), and $W_{w_m}=\{E_m^1,E_m^2...,E_m^k\}$,
$W_{w_n}=\{E_n^1,E_n^2,...,E_n^k\}$. Then for each entity $E_m^i$ in $W_{w_m}$, we will get learnt embedding vector
$\overrightarrow E_m^i$.
Note that, there might be different number of entities associated with the given
word. We just consider the top-$k$ entities in each entities set. An analysis of the impact of $K$ is
given in section of experiments.

For a word pair ($w_1$, $w_2$), when we compute semantic relatedness between their corrspoding entities, we
will get $k*k$ relateness results where $k$ is number of entities queried from knowledge graph.
The task which cmobines multiple semantic measurement is similar with the work in \cite{acl/IacobacciPN15}.
The author captures the different meanings of a word and transforms word embeddings to the sense level.
They utilize the weighted conbination of comparison in sense leval which achieve a high correlation coefficient.
In our work, the multiple-pair entities produce different semantic measurement. Traditional work only consider the
the relatedness measurement of closest objects. However, in this way, the contributions of the other entities would be ignored.
Following the work in \cite{acl/IacobacciPN15}, we combine all those relatedness
results reasonably to get more human-like measurement.

1) For two entities $E_1$ and $E_2$, we utilize cosine function to compute the
distance between two embedding vectors($\overrightarrow E_1$, $\overrightarrow E_2$):

\begin{small}
    \begin{equation}
        \label{cos}
        \nonumber
        Dis(\overrightarrow E_1,\overrightarrow E_2) = \frac{\overrightarrow E_1 \cdot 
        \overrightarrow E_2}{\left \| \overrightarrow E_1 \right \|\left \| \overrightarrow E_2 \right \|}
    \end{equation}
\end{small}

2) Following \cite{acl/IacobacciPN15}, we take two strategies for computing semantic relatedness of given
words $w_1$ and $w_2$ for comparison. One is conventional approach \cite{BudanitskyH06} which considers just the closest entities
amonge multiple entities vector pairs. There are $W_{w_1}$ and $W_{w_2}$ represent two entities sets associated with two
input words $w_1$ and $w_2$. $\overrightarrow E_i$ is the learnt embedding vector of entity $E_i$. We can get the
formalization for this approach.

\begin{small}
    \begin{equation}
        \label{cos}
        \nonumber
        Rel_{closest}(w_1, w_2) = \max \limits_{\substack{E_1 \in W_{w_1} \\ E_2 \in W_{w_1}}}
        Cos(\overrightarrow E_1,\overrightarrow E_2)
    \end{equation}
\end{small}

However this relatedness measure approach misses the other entities that also contribute to the computing of relatedness.
In fact, psychological studies suggest that humans, while comparing similarity between a pair of words, consider
different meanings of two words but not only the closest pairs\cite{Tversky77}. This claim can be directly popularized to computing
relatedness between words. There usually are various meanings for a single word. Analogously we can query multiple
entities associated with the word from knowledge graph. Only consider the entities pair which have the closest distance
woule not be conforming with the way of human thinking.

Therefore, there is another strategy for computing semantic relatedness, called $weighted$, in which
different entities associated with words contribute to relatedness measurement. The contributions are
scaled according to how they relative to the words. Fortunately, the lookup services of DBPedia in which can
return a list of ranked DBpedia resources for a search string or word. There is a specific label called
$Refcount$ that count the number of Wikipedia page inlinks for a resource. This number is required for ranking.
We can utilize this number to estimate the dominance of each specific entity resource in knowledge graph.
To this end, we use a operation of normalization. For each entities $E \in W_{w_i}$, we get the dominance of
$E$ by dividing the value of $Refcount$ by the sum of all $Refcount$ value of entities in  $W_{w_i}$:

\begin{small}
    \begin{equation}
        \label{cos}
        \nonumber
        d(E) = \frac{refcount(E)}{\sum_{{E}'\in W_i} refcount({E}')}
    \end{equation}
\end{small}

Besides, those entities which are closer would play a more important role in computing relatedness measurement.
Following the \cite{acl/IacobacciPN15}, we model this by biasing the relatedness computation towards closer entities through a power function with
parameter $\alpha$. The relatedness of a pair of words $w_1$ and $w_2$ is computed as using $weighted$ strategy:

\begin{small}
    \begin{equation}
        \begin{split}
        \label{cos}
        \nonumber
        Rel_{weight}&(w_1, w_2)=\\ 
        &\sum_{E_1 \in W_1}\sum_{E_2 \in W_2}d(E_1)d(E_2)Cos(\overrightarrow E_1,\overrightarrow E_2)^\alpha 
        \end{split}
    \end{equation}
\end{small}

In this strategy, we given the entity pairs which is closer a more important role to determine the final semantic relatedness
score. Experiments in below show that the $weighted$ strategy outperforms the $closest$.
% \begin{small}
%     \begin{math}
%         \label{cos}
%         \nonumber
%         Dis(E_1,E_2)
%         =\lambda Cos(\overrightarrow E_1,\overrightarrow E_2) + (1-\lambda)\frac{1}{path(E_1, E_2)}
%     \end{math}
% \end{small}


% \begin{small}
%     \begin{equation}
%         \label{cos}
%         \nonumber
%         Cos^*(\overrightarrow E_1,\overrightarrow E_2)=
%         \begin{cases} 
%         Cos(\overrightarrow E_1,\overrightarrow E_2) \times \beta, &if(s_1, s_2) \in E\\
%         Cos(\overrightarrow E_1,\overrightarrow E_2) \times \beta^{-1}, &otherwise
%         \end{cases}
%     \end{equation}
% \end{small}