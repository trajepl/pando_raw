\section{Methology}
In order to compute semantic relatedness of a pair of words, we propose a model which is
threefold. 
For a given pair of words, we first query the corresponding entities in knowledge graph.
We need to construct a special graph contains all related entities and related between the 
corrsponding entity pairs.
Then we use translating embedding for knowledge graph instead of Word2vec to train the constructed 
graph. For each entity and relationship, this method produce a representation of vector.
Besides, in the query step, we would get several corresponding entities for an input word. Inspired by
\cite{acl/IacobacciPN15}, we combine the relatedness scores resulting from the multiple pairs of entities 
as the final measure sorce in knowledge graph.

% Please add the following required packages to your document preamble:
% \usepackage{booktabs}
\begin{table*}[]
    \small
    \centering
    \caption{Query Entity}
    \label{entities}
    \begin{tabular}{@{}|l|l|l|@{}}
    \toprule
    \textbf{Words}  & automobile                                               & car                                      \\ \midrule
    \textbf{Entity} & http://dbpedia.org/resource/Automobile                   & http://dbpedia.org/resource/Automobile   \\ \midrule
    \textbf{Entity} & http://dbpedia.org/resource/Automobile                   & http://dbpedia.org/resource/NASCAR       \\ \midrule
    \textbf{Entity} & http://dbpedia.org/resource/Ferry                        & http://dbpedia.org/resource/Tram         \\ \midrule
    \textbf{Entity} & http://dbpedia.org/resource/Internal\_combustion\_engine & http://dbpedia.org/resource/Auto\_racing \\ \midrule
    \textbf{Entity} & http://dbpedia.org/resource/Gasoline                     & http://dbpedia.org/resource/Ferry        \\ \bottomrule
    \end{tabular}    
\end{table*}

\subsection{Construct graph}
Our aim is to compute the semantic relatedness between a pair of words. The process of related measure
need complete and ample background knowledge which can be gathered in knowledge graph, such as DBPedia, YAGO etc.
The first problem we face is how to obtain knowledge from knowledge graph. In our model, we utilize the DBPedia as
our knowledge base to gather corresponding entities triggered by given words. Our model relies on lookup
services such as that provided by DBPedia \footnote{http://lookup.dbpedia.org/api/search/KeywordSearch}. As show 
in Table \ref{entities}, for a given word pair \emph{automobile and car}, 
we can get the corresponding entities in DBPedia. The query returns URIs for representing entities in DBPedia. 
These URIs describe entities so accurate that we can access knowledge graph using powerful query language SPARQL 
to get everything we want.

We can use the query results access knowledge graph now. Next we need to construct graph 
contains all related entities and related between the corrsponding entity pairs. Inspired by \cite{aaai/NavigliP12},
we propose a improved method to get our semantic graph. We mark entities corresponding to
given words \emph{$w_1$} and \emph{$w_2$} as \emph{$E_1$} and \emph{$E_2$}. 
Then we start by selecting the subgraph of DBPedia which contains all the path between entities
($ENT = \emph{$E_1$} \cup \emph{$E_2$}$) and all attributes for each each entity in $ENT$. We do 
this by building a directed grah ${G = (V, E)}$ which contains all relevant information which describes
the entities set.

i) We first define the set $V$ in $G$: $V:=ENT$. 
The size of set $V$ is not fixed. It would be extended in the following steps.
As for the set $E$, we initialize it as empty, i.e., $E:=\emptyset$.

ii) In this step, we would find all path connecting the nodes in $V$. For each $v \in V$, 
we adopt Depth-Fist Search(DFS) to go through the knowledge graph. Every time we find a node
$v^, \in V$ but $v \ne v^,$ along a path($v, v_1, v_2,...,v_n, v^,$). Then we add all intermedia 
nodes and edges in this path to $G$, i.e., $V:=V \cup \{v_1, ..., v_n\}$, 
$E:=E \cup \{(v, v_1), ..., (v_k, v)\}$.

iii) Next, we get all relevant attributes which are described as literal, number or something 
else special sympol in knowledge graph. For example, there is a person A, his age is \emph{24}. He is a \emph{male}.
The number \emph{42}, the literal \emph{male} are not entities in knowledge graph,
but all are the attributes surrounding this person.
For each $v \in V$, we collect all the surrounding attributes 
$\{a_1, a_2, ..., a_k\}$. Then we have $V:=V \cup \{a_1, ..., a_k\}$, 
$E:=E \cup \{(v_i, a_1), ..., (v_i, a_k)\}$ ($v_i \in V$).

By this way, we extract a subgraph from DBPedia which consist of the all relevant information which describes
the entities set.

\subsection{Computing Semantic Relatedness}

\cite{corr/Ledell17}

\cite{aaai/BordesWCB11}


\subsection{Optimization}
\cite{acl/IacobacciPN15}